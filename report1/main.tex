\documentclass[nohyperref]{article}
\usepackage{microtype}
\usepackage{graphicx}
\usepackage{subfigure}
\usepackage{booktabs} % for professional tables

% hyperref makes hyperlinks in the resulting PDF.
% If your build breaks (sometimes temporarily if a hyperlink spans a page)
% please comment out the following usepackage line and replace
% \usepackage{icml2022} with \usepackage[nohyperref]{icml2022} above.
\usepackage{hyperref}

% Attempt to make hyperref and algorithmic work together better:
\newcommand{\theHalgorithm}{\arabic{algorithm}}

\usepackage[accepted]{icml2022}

% If accepted, instead use the following line for the camera-ready submission:
% \usepackage[accepted]{icml2022}

\usepackage{amsmath}
\usepackage{amssymb}
\usepackage{mathtools}
\usepackage{amsthm}
% if you use cleveref..
\usepackage[capitalize,noabbrev]{cleveref}
\theoremstyle{plain}
\newtheorem{theorem}{Theorem}[section]
\newtheorem{proposition}[theorem]{Proposition}
\newtheorem{lemma}[theorem]{Lemma}
\newtheorem{corollary}[theorem]{Corollary}
\theoremstyle{definition}
\newtheorem{definition}[theorem]{Definition}
\newtheorem{assumption}[theorem]{Assumption}
\theoremstyle{remark}
\newtheorem{remark}[theorem]{Remark}

% Todonotes is useful during development; simply uncomment the next line
%    and comment out the line below the next line to turn off comments
%\usepackage[disable,textsize=tiny]{todonotes}
\usepackage[textsize=tiny]{todonotes}

% The \icmltitle you define below is probably too long as a header.
% Therefore, a short form for the running title is supplied here:
% \icmltitlerunning{WebQA Team 6}

\begin{document}

    \twocolumn[
        \icmltitle{WebQA Team 6, TBD}

% It is OKAY to include author information, even for blind
% submissions: the style file will automatically remove it for you
% unless you've provided the [accepted] option to the icml2022
% package.

% List of affiliations: The first argument should be a (short)
% identifier you will use later to specify author affiliations
% Academic affiliations should list Department, University, City, Region, Country
% Industry affiliations should list Company, City, Region, Country

% You can specify symbols, otherwise they are numbered in order.
% Ideally, you should not use this facility. Affiliations will be numbered
% in order of appearance and this is the preferred way.
        \icmlsetsymbol{equal}{*}

        \begin{icmlauthorlist}
            \icmlauthor{Haofei Yu}{equal,cmu}
            \icmlauthor{Jiyang Tang}{equal,cmu}
            \icmlauthor{Ruiyi Wang}{equal,cmu}
            \icmlauthor{Ziang Zhou}{equal,cmu}
        \end{icmlauthorlist}

        \icmlaffiliation{cmu}{Team 6}

        \icmlkeywords{Machine Learning, ICML}

        \vskip 0.3in
    ]

% this must go after the closing bracket ] following \twocolumn[ ...

% This command actually creates the footnote in the first column
% listing the affiliations and the copyright notice.
% The command takes one argument, which is text to display at the start of the footnote.
% The \icmlEqualContribution command is standard text for equal contribution.
% Remove it (just {}) if you do not need this facility.

%\printAffiliationsAndNotice{}  % leave blank if no need to mention equal contribution
%\printAffiliationsAndNotice{\icmlEqualContribution} % otherwise use the standard text.

%\begin{abstract}
%\end{abstract}


    \section{Introduction}\label{intro}

    Remember to mention our github repository https://github.com/tjysdsg/MMML-Fall22


    \section{Experimental Setup}

    \subsection{WebQA Benchmark}

    Describe the benchmark + difference with others (strength)

    \subsection{Data}

    Raw data

    \subsection{Feature Extraction}

    RCNN features

    \subsection{Baselines}

    Baselines

    \subsection{Results}

    Metrics + results


    \section{Related Work}

    \subsection{Multimodal Datasets}

    \subsection{Pretraining Methods}

    \subsection{VQA}


    \section{Research Ideas}

    \subsection{1234}

    \subsection{}

    Test citation \cite{mitchell80}

    \bibliography{main}
    \bibliographystyle{icml2022}

\end{document}
